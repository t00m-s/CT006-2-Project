\documentclass{article}
% !TEX encoding = UTF-8 Unicode
% !TEX root = docs/source/main.tex

\usepackage{titlesec}
\usepackage{tcolorbox}
\usepackage[margin=1in]{geometry}
\usepackage{float}
\usepackage{graphicx}
\usepackage[export]{adjustbox}


\titleformat*{\section}{\huge\bfseries}
\titleformat*{\subsection}{\LARGE\bfseries}
\titleformat*{\subsubsection}{\large\bfseries}
\titleformat*{\paragraph}{\large\bfseries}
\titleformat*{\subparagraph}{\small\bfseries}
\newcommand{\myparagraph}[1]{\paragraph{#1}\mbox{}\\}


\usepackage{listings}


\definecolor{CPPLight}  {HTML} {686868}
\definecolor{CPPSteel}  {HTML} {888888}
\definecolor{CPPDark}   {HTML} {262626}
\definecolor{CPPBlue}   {HTML} {4172A3}
\definecolor{CPPGreen}  {HTML} {487818}
\definecolor{CPPBrown}  {HTML} {A07040}
\definecolor{CPPRed}    {HTML} {AD4D3A}
\definecolor{CPPViolet} {HTML} {7040A0}
\definecolor{CPPGray}  {HTML} {B8B8B8}
\definecolor{backcolour}{rgb}{0.95,0.95,0.92}

\tcbset{colback=white,colframe=black!10}

\lstset{
    columns=fixed,       
    numbers=left,                                        
    frame=none,                                          
    backgroundcolor=\color{backcolour},            
    keywordstyle=\color[RGB]{255,0,127},                 
    numberstyle=\footnotesize\color{darkgray},
    commentstyle=\it\color[RGB]{0,153,76},                
    stringstyle=\rmfamily\slshape\color[RGB]{128,0,0},   
    showstringspaces=false,  
    basicstyle=\ttfamily\footnotesize,                     
    language=SQL,                                        
    morekeywords={},
    emphstyle=\color{CPPViolet}, 
}

\title{Documentazione di progetto}
\author{Nicola Panizzolo, Tommaso Soncin, Michael Sarto}
\date{CT006 2023/2024}
\usepackage{tgadventor}

\setcounter{section}{-1}
\begin{document}

\lstset{inputencoding = utf8,  % Input encoding
    extendedchars = true,  % Extended ASCII
    literate      =        % Support additional characters
      {á}{{\'a}}1  {é}{{\'e}}1  {í}{{\'i}}1 {ó}{{\'o}}1  {ú}{{\'u}}1
      {Á}{{\'A}}1  {É}{{\'E}}1  {Í}{{\'I}}1 {Ó}{{\'O}}1  {Ú}{{\'U}}1
      {à}{{\`a}}1  {è}{{\`e}}1  {ì}{{\`i}}1 {ò}{{\`o}}1  {ù}{{\`u}}1
      {À}{{\`A}}1  {È}{{\'E}}1  {Ì}{{\`I}}1 {Ò}{{\`O}}1  {Ù}{{\`U}}1
      {ä}{{\"a}}1  {ë}{{\"e}}1  {ï}{{\"i}}1 {ö}{{\"o}}1  {ü}{{\"u}}1
      {Ä}{{\"A}}1  {Ë}{{\"E}}1  {Ï}{{\"I}}1 {Ö}{{\"O}}1  {Ü}{{\"U}}1
      {â}{{\^a}}1  {ê}{{\^e}}1  {î}{{\^i}}1 {ô}{{\^o}}1  {û}{{\^u}}1
      {Â}{{\^A}}1  {Ê}{{\^E}}1  {Î}{{\^I}}1 {Ô}{{\^O}}1  {Û}{{\^U}}1
      {œ}{{\oe}}1  {Œ}{{\OE}}1  {æ}{{\ae}}1 {Æ}{{\AE}}1  {ß}{{\ss}}1
      {ç}{{\c c}}1 {Ç}{{\c C}}1 {ø}{{\o}}1  {Ø}{{\O}}1   {å}{{\r a}}1
      {Å}{{\r A}}1 {ã}{{\~a}}1  {õ}{{\~o}}1 {Ã}{{\~A}}1  {Õ}{{\~O}}1
      {ñ}{{\~n}}1  {Ñ}{{\~N}}1  {¿}{{?`}}1  {¡}{{!`}}1
      {°}{{\textdegree}}1 {º}{{\textordmasculine}}1 {ª}{{\textordfeminine}}1
  }

\maketitle
\tableofcontents
\newpage

\section{Introduzione}
\subsection{Tema del progetto}
\par
Viene richiesto di curare il design e l'implementazione di una web application per la valutazione interna dei
progetti di ricerca da sottoporre per finanziamento all'Unione Europea. I ricercatori devono poter creare
nuovi progetti da sottoporre a valutazione, ciascuno dei quali comprende una descrizione testuale ed uno o
pi`u documenti in formato PDF da valutare, possibilmente di tipo diverso (data management plan, ethics
deliverable, ecc.). Ciascun progetto ha uno stato, fra cui:
\begin{itemize}
    \item approvato: i valutatori hanno espresso parere favorevole e non sono più possibili modifiche;
    \item sottomesso per valutazione: sottoposto per la prossima finestra di valutazione, ma non ancora valutato;
    \item richiede modifiche: i valutatori hanno richiesto modifiche in vista di un'ulteriore valutazione;
    \item non approvato: i valutatori hanno espresso parere contrario e non sono più possibili modifiche;
\end{itemize}
I valutatori possono accedere ai diversi progetti da valutare nella prossima finestra di valutazione, scaricare
i documenti da valutare e creare un report di valutazione per ognuno di essi, oltre ad aggiornare lo stato del
progetto, per esempio ad “approvato”. I ricercatori hanno accesso ai report di valutazione e possono rivedere
i loro progetti alla luce degli stessi, finchè non vengono approvati o rifiutati.\\\\Vengono forniti alcuni spunti possibili per arricchire lo scenario, senza pretesa di esaustività:
\begin{itemize}
    \item Ogni progetto ha uno storico di versione per tenere traccia delle diverse iterazioni del processo di
    valutazione: in particolare, per ogni documento del progetto, i ricercatori ed i valutatori devono poter
    accedere a tutte le versioni sottomesse precedentemente ed al relativo report di valutazione. Le versioni
    possono includere informazioni ausiliarie, che dettaglino i cambiamenti rispetto alle versioni precedenti.
    \item Ogni progetto include una componente di messaggistica, tramite la quale i ricercatori ed i valutatori
    possono interagire. Per esempio i ricercatori possono chiedere ulteriori chiarimenti riguardo ad una
    valutazione ricevuta, a cui i valutatori possono rispondere in maniera anonima.
    \item I valutatori possono inserire commenti puntuali direttamente all'interno dei PDF sottomessi per valutazione, per esempio in forma di note. In questo modo i report di valutazione possono fare riferimento
    anche a tali commenti puntuali (es. “si veda la nota a pagina 4").
    
\end{itemize}


\subsection{Descrizione}
L'applicazione è una piattaforma completa per la gestione della valutazione dei progetti di ricerca. L'obiettivo principale di questa piattaforma è fornire ai ricercatori uno strumento efficiente per la valutazione dei progetti.\\\\Una delle caratteristiche chiave dell'applicazione è la possibilità di inserire i propri progetti all'interno del sistema, consentendo ad altri utenti autorizzati di eseguire la valutazione.\\\\Ogni progetto ha più stati visibili all'autore e a tutti gli utenti autorizzati, in modo da tenere traccia dell'avanzamento della valutazione.\\\\La piattaforma offre una serie di funzionalità  per facilitare il processo di valutazione, come la possibilità di scrivere note e caricare correzioni.\\\\L'obiettivo finale dell'applicazione è migliorare l'efficienza nel processo di valutazione dei progetti di ricerca, fornendo una soluzione completa e intuitiva per gli utenti.
\section{Funzionalità principali}
Di seguito forniremo una breve descrizione delle principali funzionalità implementate nell'applicazione. 
\subsection{Gestione degli account}
\subsubsection{Registrazione}
L'utente ha la possibilità di registrarsi autonomamente tramite un form in cui inserire i propri dati ed effettuare la creazione della password, inizialmente tutti gli utenti sono registrati come ricercatori, se dovesse essere necessaria la modifica di un ruolo sarà compito dell'amministratore;

\subsubsection{Login}
Dopo aver effettuato la registrazione rispettando i vari criteri per mail e password l'utente potrà effettuare il login all'interno dell'applicazione. 

\begin{figure}[!h]
    \begin{minipage}[c]{0.5\linewidth}
        \centering
        \fbox{\includegraphics[width=0.5\textwidth]{registrazione}}\\
        \caption{\raggedleft Screenshot registrazione}
    \end{minipage}\hfill
    \begin{minipage}[c]{0.5\linewidth}
        \centering
        \fbox{\includegraphics[width=0.7\textwidth]{login}}\\
        \caption{Screenshot login}
    \end{minipage}
\end{figure}

\subsection{Gestione dei progetti}
\subsubsection{Caricamento di un progetto} 
L'utente ha la possibilità di caricare un progetto allegando multipli file $.pdf$, per concludere l'operazione è necessario selezionare il tipo di progetto da un menù a tendina, inserire una breve descrizione e facoltativamente scrivere una nota a riguardo.
\subsubsection{Creazione degli stati per la valutazione}
I possibili stati di un progetto sono rimasti gli stessi descritti nel tema del progetto(Submitted, Require Changes, Approved, Not Approved), nella visualizzazione di un progetto l'utente può osservare ogni stato di tale progetto, al quale vengono affiancate 3 informazioni: i file allegati, le note del valutatore e la data.
\subsubsection{Valutazione dei progetti}
Un utente valutatore può cambiare lo stato dei progetti aggungendone uno nuovo, per farlo deve caricare uno o più file $.pdf$, così da poter mantenere il file originale e magari aggiungerne un altro con le eventuali correzioni, dopo aver caricato i file può selezionare il nuovo stato da un menù a tendina e aggiungere delle eventuali note, come indicato dal tema del progetto. 
\subsubsection{Chat}
In ogni progetto è presente una chat sulla quale possono scrivere il creatore del progetto e gli eventuali valutatori.
\begin{figure}[!h]
    \begin{minipage}[c]{0.5\linewidth}
        \centering
        \fbox{\includegraphics[width=0.7\textwidth]{nuovo}}\\
    \end{minipage}\hfill
    \begin{minipage}[c]{0.5\linewidth}
        \centering
        \fbox{\includegraphics[width=0.7\textwidth]{valuta}}\\
    \end{minipage}
    \caption{Pagine per la creazione e la valutazione di un progetto}
\end{figure}

\begin{figure}[!h]
    \begin{minipage}[c]{0.5\linewidth}
        \centering
        \fbox{\includegraphics[width=0.7\textwidth]{stati}}\\
        \caption{Esempio di visualizzazione degli stati di un progetto}
    \end{minipage}\hfill
    \begin{minipage}[c]{0.5\linewidth}
        \centering
        \fbox{\includegraphics[width=0.7\textwidth]{chat}}\\
        \caption{Chat di un progetto}
    \end{minipage}
\end{figure}


\subsection{Amministrazione}
\subsubsection{Ruoli}
L'assegnazione dei privilegi agli utenti dell'applicazione si basa su 3 ruoli: 
\begin{itemize}
    \item Researcher: è il ruolo con meno privilegi, può solo creare i propri progetti senza valutarne altri;
    \item Reviewer: è un Researcher che ha anche la possibilità di valutare i progetti degli altri utenti;
    \item Admin: può accedere alla Admin Dashboard e può anche valutare i progetti, questo ruolo viene assegnato dall'amministratore di sistema.
\end{itemize}
\subsubsection{Admin Dashboard}
Questa pagina consiste in una tabella interattiva con la quale l'utente admin può avere accesso ai dati anagrafici dei singoli utenti, modificarne il ruolo ed eventualmente rimuovere gli utenti dal sistema.
\begin{figure}[!h] 
    \centering
    \fbox{\includegraphics[width=0.9\textwidth]{admin}}\\
    \caption{Admin Dashboard}
\end{figure}

\section{Progettazione concettuale e logica della base di dati}

\section{Query principali}

\section{Principali scelte progettuali}

\section{Ulteriori informazioni}

\section{Contributo al progetto}

\end{document}